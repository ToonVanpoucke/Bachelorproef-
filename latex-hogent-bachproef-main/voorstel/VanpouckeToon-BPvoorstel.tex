%==============================================================================
% Sjabloon onderzoeksvoorstel bachproef
%==============================================================================
% Gebaseerd op document class `hogent-article'
% zie <https://github.com/HoGentTIN/latex-hogent-article>

% Voor een voorstel in het Engels: voeg de documentclass-optie [english] toe.
% Let op: kan enkel na toestemming van de bachelorproefcoördinator!
\documentclass{hogent-article}

% Invoegen bibliografiebestand
\addbibresource{voorstel.bib}

% Informatie over de opleiding, het vak en soort opdracht
\studyprogramme{Professionele bachelor toegepaste informatica}
\course{Bachelorproef}
\assignmenttype{Onderzoeksvoorstel}
% Voor een voorstel in het Engels, haal de volgende 3 regels uit commentaar
% \studyprogramme{Bachelor of applied information technology}
% \course{Bachelor thesis}
% \assignmenttype{Research proposal}

\academicyear{2024-2026} % TODO: pas het academiejaar aan

% TODO: Werktitel
\title{Slimme Navigatiehulp voor blinden: Verbetering van Real-Time Positie en Obstakelherkenning}

% TODO: Studentnaam en emailadres invullen
\author{Toon Vanpoucke}
\email{toon.vanpoucke@student.hogent.be}

% TODO: Medestudent
% Gaat het om een bachelorproef in samenwerking met een student in een andere
% opleiding? Geef dan de naam en emailadres hier
% \author{Yasmine Alaoui (naam opleiding)}
% \email{yasmine.alaoui@student.hogent.be}

% TODO: Geef de co-promotor op
\supervisor[Co-promotor]{n.v.t.}

% Binnen welke specialisatierichting uit 3TI situeert dit onderzoek zich?
% Kies uit deze lijst:
%
% - Mobile \& Enterprise development
% - AI \& Data Engineering
% - Functional \& Business Analysis
% - System \& Network Administrator
% - Mainframe Expert
% - Als het onderzoek niet past binnen een van deze domeinen specifieer je deze
%   zelf
%
\specialisation{AI \& Data Engineering}
\keywords{Scheme, World Wide Web, $\lambda$-calculus}

\begin{document}

\begin{abstract}
  Dit bachelorproefvoorstel richt zich op de ontwikkeling van een draagbare AI-navigatiehulp (Assistive Technology) om de zelfstandige mobiliteit van blinden in stedelijke omgevingen te verbeteren. Het belangrijkste knelpunt is de onnauwkeurigheid van GPS in stedelijke gebieden en het niet detecteren van kleine obstakels. De centrale vraag is hoe we real-time positiebepaling (VIO) en objectherkenning (Edge AI) kunnen realiseren met een extreem hoge nauwkeurigheid en snelheid op draagbare hardware. De aanpak omvat experimentele AI-optimalisatie (TensorRT) en het vergelijken van geavanceerde fusie-algoritmen (FGO versus EKF). De verwachte conclusie is dat deze geoptimaliseerde technologieën een technisch haalbaar en zeer bruikbaar PoC opleveren.
\end{abstract}

\tableofcontents

% De hoofdtekst van het voorstel zit in een apart bestand, zodat het makkelijk
% kan opgenomen worden in de bijlagen van de bachelorproef zelf.
%---------- Inleiding ---------------------------------------------------------

\section{Inleiding}%
\label{sec:inleiding}

Deze bachelorproef valt onder Assistive Technology en Edge AI. De doelgroep bestaat uit visueel gehandicapte personen en IT-ontwikkelaars van hulpmiddelen. Het concrete probleem is dat traditionele hulpmiddelen (stok, GPS) onveilig zijn voor 'last-few-meters wayfinding' in complexe stedelijke omgevingen door GPS-drift en het missen van fijne obstakels. Wij ontwikkelen een lichtgewicht, hands-free systeem (bril/camera) dat real-time omgevingsanalyse en nauwkeurige lokalisatie combineert op een kleine, embedded computer.

%---------- Stand van zaken ---------------------------------------------------

\section{Literatuurstudie}%
\label{sec:literatuurstudie}

De literatuur benadrukt twee technische knelpunten: nauwkeurige lokalisatie en efficiënte Edge AI. Voor nauwkeurigheid in de stad is Factor Graph Optimization (FGO) robuuster dan EKF. FGO maakt ook fusie met Pedestrian Dead Reckoning (PDR) mogelijk. Voor snelheid is INT8-kwantisering via TensorRT de beste methode om het YOLOv8n-model snel genoeg te maken op de Jetson Orin Nano. Cruciaal voor de gebruiker is non-visuele feedback: haptiek is ideaal voor dringende waarschuwingen, terwijl ruimtelijke audio de omgevingslay-out overbrengt.

% Voor literatuurverwijzingen zijn er twee belangrijke commando's:
% \autocite{KEY} => (Auteur, jaartal) Gebruik dit als de naam van de auteur
%   geen onderdeel is van de zin.
% \textcite{KEY} => Auteur (jaartal)  Gebruik dit als de auteursnaam wel een
%   functie heeft in de zin (bv. ``Uit onderzoek door Doll & Hill (1954) bleek
%   ...'')

%---------- Methodologie ------------------------------------------------------
\section{Methodologie}%
\label{sec:methodologie}

De methodologie is opgedeeld in vier fasen. We starten met een gebruikersstudie om de vereisten en de non-visuele feedbackstrategie te bepalen. Vervolgens optimaliseren we de Embedded AI-software (YOLOv8n) voor real-time snelheid op de hardware en valideren we de robuuste sensorfusie-algoritmen (VIO) om een hoge lokalisatienauwkeurigheid te garanderen. Dit leidt tot de finale integratie van het Proof-of-Concept, gevolgd door technische benchmarking en een empirische validatie van de gebruikerservaring.

%---------- Verwachte resultaten ----------------------------------------------
\section{Verwacht resultaat, conclusie}%
\label{sec:verwachte_resultaten}

De combinatie van geoptimaliseerde Edge AI en Factor Graph Optimization-gebaseerde VIO levert een technisch haalbaar en bruikbaar hulpmiddel op dat de onafhankelijke mobiliteit in complexe stedelijke omgevingen significant en meetbaar kan verbeteren.



\printbibliography[heading=bibintoc]

\end{document}
