%---------- Inleiding ---------------------------------------------------------

\section{Inleiding}%
\label{sec:inleiding}

Deze bachelorproef valt onder Assistive Technology en Edge AI. De doelgroep bestaat uit visueel gehandicapte personen en IT-ontwikkelaars van hulpmiddelen. Het concrete probleem is dat traditionele hulpmiddelen (stok, GPS) onveilig zijn voor 'last-few-meters wayfinding' in complexe stedelijke omgevingen door GPS-drift en het missen van fijne obstakels. Wij ontwikkelen een lichtgewicht, hands-free systeem (bril/camera) dat real-time omgevingsanalyse en nauwkeurige lokalisatie combineert op een kleine, embedded computer.

%---------- Stand van zaken ---------------------------------------------------

\section{Literatuurstudie}%
\label{sec:literatuurstudie}

De literatuur benadrukt twee technische knelpunten: nauwkeurige lokalisatie en efficiënte Edge AI. Voor nauwkeurigheid in de stad is Factor Graph Optimization (FGO) robuuster dan EKF. FGO maakt ook fusie met Pedestrian Dead Reckoning (PDR) mogelijk. Voor snelheid is INT8-kwantisering via TensorRT de beste methode om het YOLOv8n-model snel genoeg te maken op de Jetson Orin Nano. Cruciaal voor de gebruiker is non-visuele feedback: haptiek is ideaal voor dringende waarschuwingen, terwijl ruimtelijke audio de omgevingslay-out overbrengt.

% Voor literatuurverwijzingen zijn er twee belangrijke commando's:
% \autocite{KEY} => (Auteur, jaartal) Gebruik dit als de naam van de auteur
%   geen onderdeel is van de zin.
% \textcite{KEY} => Auteur (jaartal)  Gebruik dit als de auteursnaam wel een
%   functie heeft in de zin (bv. ``Uit onderzoek door Doll & Hill (1954) bleek
%   ...'')

%---------- Methodologie ------------------------------------------------------
\section{Methodologie}%
\label{sec:methodologie}

De methodologie is opgedeeld in vier fasen. We starten met een gebruikersstudie om de vereisten en de non-visuele feedbackstrategie te bepalen. Vervolgens optimaliseren we de Embedded AI-software (YOLOv8n) voor real-time snelheid op de hardware en valideren we de robuuste sensorfusie-algoritmen (VIO) om een hoge lokalisatienauwkeurigheid te garanderen. Dit leidt tot de finale integratie van het Proof-of-Concept, gevolgd door technische benchmarking en een empirische validatie van de gebruikerservaring.

%---------- Verwachte resultaten ----------------------------------------------
\section{Verwacht resultaat, conclusie}%
\label{sec:verwachte_resultaten}

De combinatie van geoptimaliseerde Edge AI en Factor Graph Optimization-gebaseerde VIO levert een technisch haalbaar en bruikbaar hulpmiddel op dat de onafhankelijke mobiliteit in complexe stedelijke omgevingen significant en meetbaar kan verbeteren.

